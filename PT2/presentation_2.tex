%%%%%%%%%%%%%%%%%%%%%%%%%%%%%%%%%%%%%%%%%
% Beamer Presentation
% LaTeX Template
% Version 1.0 (10/11/12)
%
% This template has been downloaded from:
% http://www.LaTeXTemplates.com
%
% License:
% CC BY-NC-SA 3.0 (http://creativecommons.org/licenses/by-nc-sa/3.0/)
%
%%%%%%%%%%%%%%%%%%%%%%%%%%%%%%%%%%%%%%%%%

%----------------------------------------------------------------------------------------
%	PACKAGES AND THEMES
%----------------------------------------------------------------------------------------

\documentclass{beamer}

\mode<presentation> {

% The Beamer class comes with a number of default slide themes
% which change the colors and layouts of slides. Below this is a list
% of all the themes, uncomment each in turn to see what they look like.

%\usetheme{default}
%\usetheme{AnnArbor}
%\usetheme{Antibes}
%\usetheme{Bergen}
%\usetheme{Berkeley}
%\usetheme{Berlin}
%\usetheme{Boadilla}
%\usetheme{CambridgeUS}
%\usetheme{Copenhagen}
%\usetheme{Darmstadt}
%\usetheme{Dresden}
%\usetheme{Frankfurt}
%\usetheme{Goettingen}
%\usetheme{Hannover}
%\usetheme{Ilmenau}
%\usetheme{JuanLesPins}
%\usetheme{Luebeck}
\usetheme{Madrid}
%\usetheme{Malmoe}
%\usetheme{Marburg}
%\usetheme{Montpellier}
%\usetheme{PaloAlto}
%\usetheme{Pittsburgh}
%\usetheme{Rochester}
%\usetheme{Singapore}
%\usetheme{Szeged}
%\usetheme{Warsaw}

% As well as themes, the Beamer class has a number of color themes
% for any slide theme. Uncomment each of these in turn to see how it
% changes the colors of your current slide theme.

%\usecolortheme{albatross}
%\usecolortheme{beaver}
%\usecolortheme{beetle}
%\usecolortheme{crane}
%\usecolortheme{dolphin}
%\usecolortheme{dove}
%\usecolortheme{fly}
%\usecolortheme{lily}
%\usecolortheme{orchid}
%\usecolortheme{rose}
%\usecolortheme{seagull}
%\usecolortheme{seahorse}
%\usecolortheme{whale}
%\usecolortheme{wolverine}

%\setbeamertemplate{footline} % To remove the footer line in all slides uncomment this line
%\setbeamertemplate{footline}[page number] % To replace the footer line in all slides with a simple slide count uncomment this line

%\setbeamertemplate{navigation symbols}{} % To remove the navigation symbols from the bottom of all slides uncomment this line
}

\usepackage{graphicx} % Allows including images
\usepackage{booktabs} % Allows the use of \toprule, \midrule and \bottomrule in tables
\usepackage[utf8]{inputenc} %Accenti
\usepackage{algorithm,algorithmic} %pseudo
\usepackage[italian]{babel}
\usepackage{amsmath}
\usepackage{mathtools}

\graphicspath{ {./images/} } %Path delle immagini

\newcommand{\bigO}{\ensuremath{\mathcal{O}}} % O assintotica
\newcommand{\bigT}{\ensuremath{\Theta}} % Theta Assintotica
\DeclarePairedDelimiter{\ceil}{\lceil}{\rceil} % Ceil

\renewcommand{\algorithmicforall}{\textbf{foreach}} %forall -> foreach

%----------------------------------------------------------------------------------------
%	TITLE PAGE
%----------------------------------------------------------------------------------------

\title[Parallel Computation]{Approfondimento Algoritmica} % The short title appears at the bottom of every slide, the full title is only on the title page

\author{Liva Giovanni} % Your name
\institute[Udine] % Your institution as it will appear on the bottom of every slide, may be shorthand to save space
{
Università di Udine \\ % Your institution for the title page
\medskip
%\textit{liva.giovanni@spes.uniud.it} % Your email address
}
\date{\today} % Date, can be changed to a custom date

\begin{document}

\begin{frame}
\titlepage % Print the title page as the first slide
\end{frame}

\begin{frame}
\frametitle{Argomenti} % Table of contents slide, comment this block out to remove it
\tableofcontents % Throughout your presentation, if you choose to use \section{} and \subsection{} commands, these will automatically be printed on this slide as an overview of your presentation
\end{frame}

%----------------------------------------------------------------------------------------
%	DEBUG SLIDES
%----------------------------------------------------------------------------------------
\section{Introduzione}

\begin{frame}
	\frametitle{Introduzione} % Table of contents slide, comment this block out to remove it
	Vari modelli di calcolo, si differenziano sul livello di astrazione dal modello reale
	\begin{itemize}
		\item $VLSI$: Attenzione ai limiti fisici dei processori
		\item $PRAM$: Modello teorico non implementabile che non considera i problemi di comunicazione
		\item $Circuiti$: Lo stesso modello visto a lezione può essere usato per lo studio di algoritmi paralleli
	\end{itemize}
\end{frame}


\subsection{Principio di Brent}
\begin{frame}
	\frametitle{Principio di Brent} % Table of contents slide, comment this block out to remove it
	Abbiamo una computazione parallela eseguita in $t$ passi.
	\begin{itemize}
		\item Supponiamo di avere $x_i$ operazioni ad ogni passo $i$
		\item Il numero di processori necessario è $d = max\ x_i$
		\item Possediamo solo $p<m$ processori
		\item Possiamo simulare la computazione originaria con $p$ processori in $\ceil{x_i / p}$ passi ad ogni $i$-esimo step
		\item In totale la simulazione viene eseguita in $\ceil{ \frac{\sum_i x_i}{p}    } + t$ passi
	\end{itemize}
\end{frame}


\section{Modello $PRAM$}
\begin{frame}
	\frametitle{Parallel Random-Access Machine} % Table of contents slide, comment this block out to remove it
	\begin{itemize}
		\item Si basa sul modello delle $RAM$ introducendo una memoria globale (\emph{registro accumulatore}) dove le varie $RAM$ possono comunicare
		\item In tempo $\bigO{(1)}$ possiamo r/w una cella della memoria locale o globale oppure eseguire una operazione $RAM$
	\end{itemize}
	Un programma $PRAM$ $\mathbb{P} = (\Pi_1,\dots,\Pi_q)$ è fatto da $q$ macchine $RAM$ indipendenti dove $q$ è una funzione $q(m,n)$ dove $m = |I|$ ed $n= \ell(I)$.\\
	Normalmente il numero di $RAM$ richiesto dipende solo da $m$.
\end{frame}

\subsection{Tipologie e Ugualianze}
\begin{frame}
	\frametitle{Parallel Random-Access Machine :: Tipologie e Ugualianze} % Table of contents slide, comment this block out to remove it
	In base a come gestiamo i conflitti di r/w sul registro accumulatore abbiamo 3 diverse tipologie di PRAM
	\begin{itemize}
		\item Exclusive-Read Exclusive-Write ($EREW$) 
		\item Concurrent-Read Exclusive-Write ($CREW$)
		\item Concurrent-Read Concurrent-Write ($CRCW$)
	\end{itemize}
	Per risolvere i conflitti di scrittura abbiamo 3 metodi:
	\begin{itemize}
		\item $Common$: Tutti i processi che insistono in una stessa locazione devono scrivere lo stesso valore 
		\item $Arbritary$: Tra tutti i processi che provano a scrivere, solo uno ha successo. L'algoritmo deve comunque funzionare a prescindere da chi vince
		\item $Priority$: Il processo l'identificativo più basso è quello che scrive
	\end{itemize}
\end{frame}

\begin{frame}
	\frametitle{Parallel Random-Access Machine :: Tipologie e Ugualianze} % Table of contents slide, comment this block out to remove it
	L'ordine con il quale sono state presentate è anche l'ordine di potenza dei vari modelli.\\
	Tra loro però sono correlati da un fattore logaritmico. Per cui il modello più potente, $CRCW\ Priority$ può essere simulato da una $EREW$ con lo stesso numero di processori e con un tempo parallelo aumentato di $\bigO{(log\ P)}$; dove $P$ è il numero di processori.
	\begin{itemize}
		\item Dimostrazione a voce
	\end{itemize}
\end{frame}

\subsection{Algoritmo Ottimale}

\begin{frame}
	\frametitle{Parallel Random-Access Machine :: Algoritmo Ottimale} % Table of contents slide, comment this block out to remove it
	
	\begin{block}{Definizione}
	$polylog(n) = \bigcup_{k>0}\bigO{(log^k n)}$\\
	Sia $S$ un programma sequenziale che opera in tempo $T(n)$.\\
	\end{block}
	Diremo che il programma $A$ di una $PRAM$ per $S$ che opera in tempo $t(n)$ con $p(n)$ processori è ottimale se:
	\begin{itemize}
		\item $t(n) = polylog(n)$
		\item $w(n) = p(n)\times t(n) = \bigO{(T(n))}$
	\end{itemize}
	Identifichiamo il lavoro eseguito da una $PRAM$ con la coppia $(w(t),t(n))$
\end{frame}

\subsection{Classe $NC$}
\begin{frame}
	\frametitle{Parallel Random-Access Machine :: Classe $NC$} % Table of contents slide, comment this block out to remove it
	Possiamo definire una classe per gli algoritmi paralleli chiamata $NC$ dove:
	\begin{equation}
		NC = \bigcup_{k>0} CRCW(n^k,\log^k n))
	\end{equation}
	e dal fatto che i vari modelli di $PRAM$ sono correlati da un fattore logaritmico: 
	\begin{equation}
	NC = \bigcup_{k>0} PRAM(n^k,\log^k n))
	\end{equation}
	Vengono anche definite sotto-classi di $NC$ nel seguente modo:
	\begin{equation}
	NC_j = \bigcup_{k>0} PRAM(n^k,\log^j n))
	\end{equation}
	Rimane ancora aperto il problema $NC \stackrel{?}{=} P$
	
\end{frame}


\section{Modello Circuiti}
\subsection{Definizioni}
\begin{frame}
	\frametitle{Circuiti :: Definizioni} % Table of contents slide, comment this block out to remove it
	
	\begin{block}{Definizione}
		Un circuito è un grafo etichettato aciclico (DAG). Le etichette dei nodi possono essere input, costanti, AND, OR, NOT oppure output.\\ 
		Input e costanti hanno 0 archi in ingresso, output e NOT 1, mentre AND e OR 2. I nodi di output hanno 0 archi in uscita.
	\end{block}
	\begin{block}{Definizione}
		Sia $B = \{0,1\}$. Un circuito con $n$ input e $m$ output calcola la funzione $f\colon \mathbb{B}^n\to\mathbb{B}^m$. Si suppone che l'input sia ordinato e così l'output.
	\end{block}
	\begin{block}{Definizione}
		La \emph{profondità} di un circuito è la lunghezza del più lungo cammino da un nodo di input ad uno di output.\\
		La $dimensione$ di un circuito è il numero di nodi che lo compongono 
	\end{block}
	
\end{frame}

\begin{frame}
	\frametitle{Circuiti :: Definizioni} % Table of contents slide, comment this block out to remove it
	
	\begin{block}{Definizione}
		Una famiglia di circuiti $C = \{C_i\}, i=1,2,\dots$ dove ogni $C_i$ ha esattamente $i$ input e $\ell(i)$ output.\\
		La famiglia $C$ di circuiti risolve un problema $P$ se la funzione compiuta da $C_i$ produce la stessa trasduzione di $P$ sulla stringa di input lunga $i$
	\end{block}
	\begin{block}{Definizione}
		Una famiglia di circuiti $C$ è \emph{logspace uniforme} se il circuito $C_n$ può essere generato da una macchina di Turing in spazio $\bigO{(\log n)}$
	\end{block}
	\begin{block}{Definizione classe CKT}
		Per $C(n) \geq n$ e $D(n) \geq n$ indichiamo con $CKT(C(n),D(n))$ la classe dei circuiti $C = \{C_n\}$ \emph{logspace uniforme} che risolve il problema $P$ con dimensione $\bigO{(C(n))}$ e profondità $\bigO{(D(n))}$
	\end{block}
\end{frame}


\subsection{Relazioni con $PRAM$}

\begin{frame}
	\frametitle{Circuiti :: Unbounded} % Table of contents slide, comment this block out to remove it
	\begin{block}{Definizione}
		$Unbounded\ fan-in\ circuit $: rimuoviamo la restrizione sul numero di input per i nodi AND e OR\\
		Hanno una correlazione polinomiale con i cricuiti classici, possiamo convertire ogni nodo con un albero fatto da nodi con 2 input
	\end{block}
	\begin{itemize}
		\item Come per i circuiti classici possiamo dare la definizione della Classe UCKT
		\item Se $P \in UCKT(p(n),d(n)) \Longrightarrow P \in CKT(p(n),d(n) \times \log p(n))$
	\end{itemize}
\end{frame}

\begin{frame}
	\frametitle{Circuiti :: Relazioni con $PRAM$} % Table of contents slide, comment this block out to remove it
	Dato un circuito $C \in UCKT(S(n),D(n))$ lo possiamo simulare con una $CRCW$ in tempo $\bigO{(D(n))}$ e con $S(n)$ processori usando $n + M(n)$ locazioni di memoria.
	\begin{block}{Dimostrazione}
		\begin{itemize}
			\item Portiamo i nodi NOT sugli input (cheat)
			\item I processori corrispondono agli archi del grafo, le locazioni di memoria ai nodi
			\item L'input è dalla locazione 1 alla $n$, le locazioni $n+i$ con $i\in\{1\dots M(n)\}$ sono inizializzate con 0 se OR e 1 se AND
			\item Ad ogni passo, il processore $p$ determina il valore corrente $c$ sull'arco $e=(u,v)$ leggendo la locazione $n+u$
			\item Se $c=0 \wedge v=$  OR, oppure $c=1 \wedge v=$ AND $\Longrightarrow$ scrive il valore di $c$ nella locazione $n+v$ eseguendo la relativa operazione
			\item Se $c=1 \wedge v=$  OR, oppure $c=0 \wedge v=$ AND $\Longrightarrow$ scrive il valore di $c$ nella locazione $n+v$
		\end{itemize}
	\end{block}
\end{frame}

\begin{frame}
	\frametitle{Circuiti :: Relazioni con $PRAM$} % Table of contents slide, comment this block out to remove it
	La dimostrazione inversa invece prevede di fornire un circuito per ogni operazione di una RAM.
	I circuiti, per come si è potuto vedere durante il laboratorio di programmazione, sono tutti a profondità costante.
	Ne consegue quindi:
	\begin{equation}
		CRCW(P(n),T(n)) \subseteq UCKT(poly(P(n)), T(n))
	\end{equation}
	Dove $poly(n) = \bigcup_{k>0} n^k$
\end{frame}

\section{Macchine di Turing Alternate}
\begin{frame}
	\frametitle{Macchine di Turing Alternate} % Table of contents slide, comment this block out to remove it
	\begin{itemize}
		\item Derivano dalla teoria degli automi alternati
		\item Sono una generalizzazione delle MdT non deterministiche
		\item Ogni stato è esistenzione oppure universale
	\end{itemize}
	A partire da uno configurazione $\alpha$ possiamo dire che è di accetazione se:
	\begin{itemize}
		\item Se $\alpha$ è \textbf{esistenziale} allora esiste una computazione che porta ad una configurazione $yes$
		\item Se $\alpha$ è \textbf{universale} allora tutte le computazioni a partire da $\alpha$ arrivano ad una configurazione $yes$
	\end{itemize}
	Si dimostra che: $ATM(S(n),T(n)) \subseteq CKT(c^{S(n)},T(n))$ con $ATM(S(n),T(n))$ la classe delle macchine di Turing alternate che operano in spazio $\bigO{(S(n))}$ e tempo $\bigO{(T(n))}$.\\
	Un ulteriore risultato è: $NC = ATM(\log n,polylog(n))$

	
\end{frame}

\section{Problemi}
\subsection{Matrix Multiplication}
\subsection{Reachability}
\subsection{Prefix Sum}

\begin{frame}
\frametitle{Paragraphs of Text}
\end{frame}



%----------------------------------------------------------------------------------------
%	PRESENTATION SLIDES
%----------------------------------------------------------------------------------------

%%------------------------------------------------
%\section{First Section} % Sections can be created in order to organize your presentation into discrete blocks, all sections and subsections are automatically printed in the table of contents as an overview of the talk
%%------------------------------------------------
%
%\subsection{Subsection Example} % A subsection can be created just before a set of slides with a common theme to further break down your presentation into chunks
%
%\begin{frame}
%\frametitle{Paragraphs of Text}
%Sed iaculis dapibus gravida. Morbi sed tortor erat, nec interdum arcu. Sed id lorem lectus. Quisque viverra augue id sem ornare non aliquam nibh tristique. Aenean in ligula nisl. Nulla sed tellus ipsum. Donec vestibulum ligula non lorem vulputate fermentum accumsan neque mollis.\\~\\
%
%Sed diam enim, sagittis nec condimentum sit amet, ullamcorper sit amet libero. Aliquam vel dui orci, a porta odio. Nullam id suscipit ipsum. Aenean lobortis commodo sem, ut commodo leo gravida vitae. Pellentesque vehicula ante iaculis arcu pretium rutrum eget sit amet purus. Integer ornare nulla quis neque ultrices lobortis. Vestibulum ultrices tincidunt libero, quis commodo erat ullamcorper id.
%\end{frame}
%
%%------------------------------------------------
%
%\begin{frame}
%\frametitle{Bullet Points}
%\begin{itemize}
%\item Lorem ipsum dolor sit amet, consectetur adipiscing elit
%\item Aliquam blandit faucibus nisi, sit amet dapibus enim tempus eu
%\item Nulla commodo, erat quis gravida posuere, elit lacus lobortis est, quis porttitor odio mauris at libero
%\item Nam cursus est eget velit posuere pellentesque
%\item Vestibulum faucibus velit a augue condimentum quis convallis nulla gravida
%\end{itemize}
%\end{frame}
%
%%------------------------------------------------
%
%\begin{frame}
%\frametitle{Blocks of Highlighted Text}
%\begin{block}{Block 1}
%Lorem ipsum dolor sit amet, consectetur adipiscing elit. Integer lectus nisl, ultricies in feugiat rutrum, porttitor sit amet augue. Aliquam ut tortor mauris. Sed volutpat ante purus, quis accumsan dolor.
%\end{block}
%
%\begin{block}{Block 2}
%Pellentesque sed tellus purus. Class aptent taciti sociosqu ad litora torquent per conubia nostra, per inceptos himenaeos. Vestibulum quis magna at risus dictum tempor eu vitae velit.
%\end{block}
%
%\begin{block}{Block 3}
%Suspendisse tincidunt sagittis gravida. Curabitur condimentum, enim sed venenatis rutrum, ipsum neque consectetur orci, sed blandit justo nisi ac lacus.
%\end{block}
%\end{frame}
%
%%------------------------------------------------
%
%\begin{frame}
%\frametitle{Multiple Columns}
%\begin{columns}[c] % The "c" option specifies centered vertical alignment while the "t" option is used for top vertical alignment
%
%\column{.45\textwidth} % Left column and width
%\textbf{Heading}
%\begin{enumerate}
%\item Statement
%\item Explanation
%\item Example
%\end{enumerate}
%
%\column{.5\textwidth} % Right column and width
%Lorem ipsum dolor sit amet, consectetur adipiscing elit. Integer lectus nisl, ultricies in feugiat rutrum, porttitor sit amet augue. Aliquam ut tortor mauris. Sed volutpat ante purus, quis accumsan dolor.
%
%\end{columns}
%\end{frame}
%
%%------------------------------------------------
%\section{Second Section}
%%------------------------------------------------
%
%\begin{frame}
%\frametitle{Table}
%\begin{table}
%\begin{tabular}{l l l}
%\toprule
%\textbf{Treatments} & \textbf{Response 1} & \textbf{Response 2}\\
%\midrule
%Treatment 1 & 0.0003262 & 0.562 \\
%Treatment 2 & 0.0015681 & 0.910 \\
%Treatment 3 & 0.0009271 & 0.296 \\
%\bottomrule
%\end{tabular}
%\caption{Table caption}
%\end{table}
%\end{frame}
%
%%------------------------------------------------
%
%\begin{frame}
%\frametitle{Theorem}
%\begin{theorem}[Mass--energy equivalence]
%$E = mc^2$
%\end{theorem}
%\end{frame}
%
%%------------------------------------------------
%
%\begin{frame}[fragile] % Need to use the fragile option when verbatim is used in the slide
%\frametitle{Verbatim}
%\begin{example}[Theorem Slide Code]
%\begin{verbatim}
%\begin{frame}
%\frametitle{Theorem}
%\begin{theorem}[Mass--energy equivalence]
%$E = mc^2$
%\end{theorem}
%\end{frame}\end{verbatim}
%\end{example}
%\end{frame}
%
%%------------------------------------------------
%
%\begin{frame}
%\frametitle{Figure}
%Uncomment the code on this slide to include your own image from the same directory as the template .TeX file.
%%\begin{figure}
%%\includegraphics[width=0.8\linewidth]{test}
%%\end{figure}
%\end{frame}
%
%%------------------------------------------------
%
%\begin{frame}[fragile] % Need to use the fragile option when verbatim is used in the slide
%\frametitle{Citation}
%An example of the \verb|\cite| command to cite within the presentation:\\~
%
%This statement requires citation \cite{p1}.
%\end{frame}
%
%%------------------------------------------------
%
%\begin{frame}
%\frametitle{References}
%\footnotesize{
%\begin{thebibliography}{99} % Beamer does not support BibTeX so references must be inserted manually as below
%\bibitem[Smith, 2012]{p1} John Smith (2012)
%\newblock Title of the publication
%\newblock \emph{Journal Name} 12(3), 45 -- 678.
%\end{thebibliography}
%}
%\end{frame}
%
%%------------------------------------------------
%
%\begin{frame}
%v
%\end{frame}
%
%%----------------------------------------------------------------------------------------

\end{document} 